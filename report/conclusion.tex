\section{Discussion}

Our failure to find results echos much the previous work on the topic of TJS. It remains to be an elusive issue that will continue to require the application of various machine learning algorithms to slowly reveal risk factors. However, our main issue from the start was curse of dimensionality. We tried to collect and analyze too much data which created confusion and no results. Given the large dataset we have already built, our data collection and cleaning processes will be made much easier for deeper and future analysis.

A limiting factor for pitch specific data is that the PITCHf/X system was only installed in MLB stadiums starting in 2008. As such, the most valuable data pertaining specifically to pitchers performance is very limited. On top of this problem is the constant updating and tweaking to the PITCHf/X system itself can create errors and variance within the data. As such it is reasonable to believe that we are still several years away from collecting enough data to make reliable predictions based on this data.

Given the small data set it another alternative would be to work on trends within the season leading up to the injury. Trends may be identified related to fatigue and overuse that may elucitate identifiably risk factors. This would be an extremely time consuming process as data would need to be pulled for every pitcher for every game and then analyzed. While no single game is likely to be of value, trends within season may be more valuable than season means as we and others have attempted to look at.

