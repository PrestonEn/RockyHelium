\section{Conclusion}

Given the lack of results from our dataset it is clear that we attempted to analyze too much data. Due to the magnitude of the dataset we collected the majority of our time was spent trying to clean and combine a massive dataset for analysis. While in theory this appeared to be a reasonable goal it appeared to be unnecessary. The much smaller dataset provided by RockFence LLC also failed to give a good model but was able to reiterate other findings regarding the importance of fastballs.

Future work should take this cue and focus on aspects of a pitchers performance that involves only high speed pitches. It may not be the velocity itself that is off importance but the frequency with which the these high speed pitches are thrown. Pitches of interest would include fastballs, cutters, and change ups. While cutters and change ups have slower velocities due to spin on the ball they may have similar issues regarding the stresses placed on the pitchers elbow as these pitchers start out fast and slow as they approach the plate.

A principal compenent analysis could be run on data only on these pitches to continue to reduce the number of components for analysis. This is likely necessary given the small number of pitchers available for sample selection. A neural network would be of extreme interest and was our initial goal when we set out to complete this project. Unfortunately the data collection and cleansing stages took up the majority of our time and left us unable to reduce the data small enough that it would have made sense to build a proper sized network as we were dealing with small sample size and large variable number.

Overall this is an interesting problem to undertake and is only now being addressed. The most recent work was only published in April 2016 so in the coming years we will hopefully have a better dataset and more accurate predictions of risk factors involved.