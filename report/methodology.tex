\section{Methodology}

\subsection{Data Collection}

Data was collected and combined from several sources available online. Standard baseball statistics such as hits, walks, strike outs, etc. were collected from the Baseball Databank which is a publically maintained dataset of historical baseball data. Data was cloned from the Chadwick Baseball Databank GitHub repository \cite{ChadwickBD}. PITCHf/X and pitch count data was scraped from http://brooksbaseball.net using the XML package in RStudio 0.99.893. Players who underwent TJS between 2010-2015 were identified in the Baseball Injury Consultants private database. Access to this database was graciously provided to us by RockFence LLC. Data was obtained from this database using MySQLWorkbench 6.3.6 build 511 CE. Players were referenced by MLB media ID for PITCHf/X and Baseball IC data and by a Baseball Bureau reference ID for stats data obtained in the Baseball Databank. In order to cross reference the different ID types the Chadwick Baseball Bureau Register was used \cite{ChadwickR}. All data collectoin was conducted on a Macbook Pro with OS X El Capitain 10.11.4.

To construct the data set first a list of potential TJS case pitchers were obtained. This was done by identifying all pitchers labelled in the Baseball IC database as having undergone UCL Reconstruction. The year of surgery was considered as the index year for that pitcher. If a pitcher had multiple TJS only the first index year was selected. For all TJS cases PITCHf/X and statistical data was obtained for the index year and one year prior. Fielding Independent Pitching (FIP) calculations were made using the equation in Figure \ref{eq:fip}. The FIPs constants were collected from FanGraphs \cite{Fangraphs}. FIP is an attempt to give a measure similar to ERA but with more emphasis on plays that the pitcher has direct control over. This attempts to give a performance measure for the pitcher while removing the effect a luck and defense has on any individual pitcher \cite{FangraphsFIP}.

\begin{figure}[hb]
    \centering
        \[ FIP = \frac{13 * HR + 3 * (BB + HBP) - 2 * K}{IP} + FIPc\]
    \caption{FIP equation: HR is home runs, BB is walks, HBP is batters hit by pitch, K is strike outs, IP is innings pitched, and FIPc is the FIP constant. \cite{FangraphsFIP}}
    \label{eq:fip}
\end{figure}

Inputs were selected based on what was used in previous literature as well as taking into account critiques and improvements \cite{Gray2014}. From the statistical data, we selected home runs (HR), walks (BB), batters hit by pitch (HBP), strike outs (K), and innings pitched (IP). As stated above, these statistical variables were used to calculate the FIP score for each pitcher in an attempt to get a comparable performance statistic between pitchers. From the PITCHf/X data we selected average release speed (mph), max release speed (maxmph), horizontal movement (pfx\_x), vertical movement (pfx\_z), horizontal pitch location (hloc), vertical pitch location (vloc), and grooved pitches (bway) which are pitches straight down the center of home plate. These variables were selected for each pitch type the player had used in the index year and year prior. Pitch counts were also collected for each pitch type a player had. The number of pitch types each pitcher has, therefore the number of variables with values could vary. To account for this we replaced any NA in pitchF/x and pitch count data with a 0 value.

Control pitchers were seleced with regard to TJS pitchers. For each index year between 2010-2015 groups were created where each control pitcher had data for that year and the year prior. For selection in analysis the TJS pitchers were first selected and then a control pitcher was selected from their respective bins based on TJS index year. We did no control for pitchers age or throwing arm, however birth year was included as an input.

\subsection{Logistic Regression}

\subsection{Neural Network}